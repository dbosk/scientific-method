\documentclass[a4paper,10pt,article,oneside]{memoir}
%%% Tufte %%%
\usepackage{marginfix}
%\setlength{\evensidemargin}{\oddsidemargin}
\marginparmargin{outer}
\setlrmarginsandblock{2.5cm}{8cm}{*}

\footnotesinmargin

\usepackage{ragged2e}
\renewcommand{\sidefootform}{\RaggedRight}
\renewcommand{\foottextfont}{\footnotesize\RaggedRight}

\setmpjustification{\RaggedRight}{\RaggedRight}

% margin figure and caption typeset ragged against text block
\setfloatadjustment{marginfigure}{\mpjustification}
\setmarginfloatcaptionadjustment{figure}{\captionstyle{\mpjustification}}

% From https://tex.stackexchange.com/a/324757/17418
% Palatino for main text and math
\usepackage[osf,sc]{mathpazo}

% Helvetica for sans serif
% (scaled to match size of Palatino)
\usepackage[scaled=0.90]{helvet}

% Bera Mono for monospaced
% (scaled to match size of Palatino)
\usepackage[scaled=0.85]{beramono}

\setlxvchars\setxlvchars
\checkandfixthelayout

\nouppercaseheads
%%% end tufte %%%
\let\subsubsection\subsection
\let\subsection\section
\let\section\chapter

\newsubfloat{figure}% Allow subfloats in figure environment

\usepackage[utf8]{inputenc}
\usepackage[T1]{fontenc}
\usepackage[british]{babel}
\usepackage{booktabs}

\usepackage[%
  natbib,
  citestyle=verbose,singletitle=false,
  style=verbose,
  maxbibnames=99,%
  isbn=false,doi=false,url=true
]{biblatex}
\addbibresource{bibliography.bib}

\usepackage[all]{foreign}
\renewcommand{\foreignfullfont}{}
\renewcommand{\foreignabbrfont}{}

\usepackage{import}

\usepackage[strict]{csquotes}
\SetCiteCommand{\autocite}
\usepackage[single]{acro}
\acsetup{cite/cmd={\autocite}}

\usepackage[noend]{algpseudocode}
\usepackage{xparse}

\let\email\texttt

\usepackage[outputdir=ltxobj]{minted}
\setminted{autogobble}

\usepackage{pythontex}
\setpythontexoutputdir{.}
\setpythontexworkingdir{..}

\usepackage{amsmath}
\usepackage{amssymb}
\usepackage{mathtools}
\usepackage{amsthm}
\usepackage{thmtools}
%\usepackage[unq]{unique}
\DeclareMathOperator{\powerset}{\mathcal{P}}

\usepackage[binary-units]{siunitx}

\usepackage{adjustbox}
\usepackage{lipsum}
\usepackage{multicol}
\usepackage{changepage}

\usepackage{didactic}
\usepackage[capitalize]{cleveref}

\crefname{enumi}{}{}

\newcommand{\LOrelate}{\label{LOrelate}
  relate the different parts of scientific method, how they relate to one 
another, contribute and not contribute to scientificity in security}

\newcommand{\LOevaluate}{\label{LOevaluate} assess, analyse, and discuss the 
quality in, and ethical aspects of, knowledge generation related to digital 
systems and in particular the security of these systems}

\newcommand{\LOapply}{\label{LOapply} apply scientific methodology to show how 
to answer issues in the cybersecurity field}

\newcommand{\LOplan}{\label{LOplan} plan and carry out assignments within given 
time frames and available resources}

\newcommand{\LOcomm}{\label{LOcomm} write short, clear and arguing texts based 
on own analysis as well as given material}




\usepackage[noamsthm,notheorems]{beamerarticle}
\setjobnamebeamerversion{slides}

\usepackage[inline]{enumitem}

\usepackage{authblk}
\let\institute\affil

\declaretheorem[numbered=unless unique,style=theorem]{theorem}
\declaretheorem[numbered=unless unique,style=definition]{definition}
\declaretheorem[numbered=unless unique,style=definition]{assumption}
\declaretheorem[numbered=unless unique,style=definition]{protocol}
\declaretheorem[numbered=unless unique,style=example]{example}
%\declaretheorem[style=definition,numbered=unless unique,
%  name=Example,refname={example,examples}]{example}
\declaretheorem[numbered=unless unique,style=remark]{remark}
\declaretheorem[numbered=unless unique,style=remark]{idea}
\declaretheorem[numbered=unless unique,style=exercise]{exercise}
\declaretheorem[numbered=unless unique,style=exercise]{question}
\declaretheorem[numbered=unless unique,style=solution]{solution}

\begin{document}
\title{%
  How do you know it's secure?
  Anonymous communication
}
\author{Daniel Bosk\thanks{%
    This material is available under the Creative Commons 
    Attribution-NonCommercial-ShareAlike (CC-BY-NC-SA) 4.0 international 
    license.
    The material was written with some aid from GitHub Copilot.
}}
\institute{%
  KTH EECS
}

\begin{frame}
  \maketitle
\end{frame}

\mode*

\begin{abstract}
  % What's the problem?
% Why is it a problem? Research gap left by other approaches?
% Why is it important? Why care?
% What's the approach? How to solve the problem?
% What's the findings? How was it evaluated, what are the results, limitations, 
% what remains to be done?

% XXX Summary
\emph{Summary:}
In this assignment we will focus on security around password policies.
It's an important topic, because for decades people everywhere has been very 
unscientific when it comes to the design and choice of password 
policies\autocite{Estes2017Aug}.

% XXX Motivation and intended learning outcomes
\emph{Intended learning outcomes:}
This assignment focuses on practice to
\begin{itemize}
  \item \LOrelate;
  \item \LOevaluate;
  \item \LOapply;
  \item \LOcomm.
\end{itemize}

% XXX Prerequisites
\emph{Prerequisites:}
\dots

\end{abstract}

\clearpage

\section{Users get routed}

\begin{frame}
  \begin{question}[Users get routed]
    \begin{itemize}
      \item What was the research question?
      \item What methods did they use?
      \item Why is that a good method?
      \item Can we use another method?
    \end{itemize}
  \end{question}
\end{frame}

\begin{frame}
  \begin{question}
    \begin{itemize}
      \item Why do we need to use empirical methods in this case?
      \item What effect does data have on validity?
    \end{itemize}
  \end{question}
\end{frame}

\section{Shadow: Running Tor in a box}

\begin{frame}
  \begin{question}[Shadow: Running Tor in a box]
    \begin{itemize}
      \item What was the research question?
      \item What methods did they use?
      \item Why is that a good method?
      \item Can we use another method?
    \end{itemize}
  \end{question}
\end{frame}

\begin{frame}
  \begin{question}
    \begin{itemize}
      \item They evaluate the quality of a research tool.
      \item What are the challenges in doing that?
    \end{itemize}
  \end{question}
\end{frame}

\section{WF with Website Oracles}

\begin{frame}
  \begin{question}[WF with Website Oracles]
    \begin{itemize}
      \item What was the research question?
      \item What methods did they use?
      \item Why is that a good method?
      \item Can we use another method?
    \end{itemize}
  \end{question}
\end{frame}

\begin{frame}
  \begin{question}
    \begin{itemize}
      \item What type of contribution do we get from this paper?
        What is the Website Oracle?
      \item How do they show WOs exist in reality?
    \end{itemize}
  \end{question}
\end{frame}

\section{Evaluating WF in Real World}

\begin{frame}
  \begin{question}[Evaluating WF in Real World]
    \begin{itemize}
      \item What was the research question?
      \item What methods did they use?
      \item Why is that a good method?
      \item Can we use another method?
    \end{itemize}
  \end{question}
\end{frame}

\begin{frame}
  \begin{question}
    \begin{itemize}
      \item They complained about the evaluation of WF in other papers.
        What was the problem?
      \item Are there any problems with this paper's approach?
    \end{itemize}
  \end{question}
\end{frame}

\end{document}
