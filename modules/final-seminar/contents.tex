\title{%
  Assessment:\\
  Designing a methodology to answer a question
}
\author{Daniel Bosk\thanks{%
   This material was authored by Daniel Bosk and is available under the 
   Creative Commons Attribution-ShareAlike (CC-BY-SA) 4.0 international 
   license.
   GitHub Copilot was used to autocomplete parts of the source code of this 
   document.
}}
\institute{%
  KTH EECS
}

\mode*

\begin{frame}
  \maketitle
\end{frame}

\begin{abstract}
  % What's the problem?
% Why is it a problem? Research gap left by other approaches?
% Why is it important? Why care?
% What's the approach? How to solve the problem?
% What's the findings? How was it evaluated, what are the results, limitations, 
% what remains to be done?

% XXX Summary
\emph{Summary:}
In this module we will try to overview the methodological state of security.

% XXX Motivation and intended learning outcomes
\emph{Intended learning outcomes:}
This assignment focuses on practice to
\begin{itemize}
  \item \LOrelate;
  \item \LOevaluate;
  \item \LOapply;
  \item \LOcomm.
\end{itemize}

% XXX Prerequisites
\emph{Prerequisites:}
We need basic knowledge of security, corresponding to an introductory course in 
the subject.
We also need a high-level overview of the breadth of research methods used in 
the area of security, corresponding to the overview lecture \enquote{The 
Scientific Method} given in the course.

\end{abstract}

\clearpage


\section{Introduction}

Now you should be able to design a method to answer a given research question.
Hence, in this assignment you should show that you can do exactly that.
You will be assigned a research question for which you should propose a way to 
answer scientifically.

\begin{frame}<presentation>
  \begin{block}{Goal}
    \begin{itemize}
      \item You should be able to evaluate a proposed method to answer a 
        question.
      \item You should be able to design a method to answer a research 
        question.
      \item Security is multifaceted, you should be able to consider more than 
        one perspective.
      \item You should be able to ask good questions and know how to answer 
        them.
    \end{itemize}
  \end{block}
\end{frame}

%Combine different methods to answer.
%Supply chain.
%Secure micro-services.
%
%Feedback to improve method for another question.
%
%Background: formal verification etc., different approaches.
%
%Inspelad presentation eller live. 4h x 4 i tentaperioden.

\section{Scenario and research question}

The research question that you should answer is the following:
\begin{frame}
\begin{restatable}{question}{mainRQ}\label{mainRQ}
  How secure is the Signal instant-messaging app?
\end{restatable}
\end{frame}

\section<article>{Assessment}\label{Assessment}

In brief, what you should be able to do is to ask good questions and propose 
suitable ways of answering them.
More formally, the learning objectives that we want to assess are the 
following; after completion of the course we want the student to be able to
\begin{enumerate}[label={(LO\arabic*)},ref=LO\arabic*]
  \item \LOrelate*;
  \item \LOevaluate*;
  \item \LOapply*;
  \item \LOplan*; and
  \item \LOcomm*.
\end{enumerate}

\subsection{Material you should produce}

To be able to show that you can do that, you should
\begin{frame}
\begin{itemize}
  \item write a report, and
  \item review someone else's report.
\end{itemize}
\end{frame}

The report should contain the following sections:
\begin{frame}
\begin{enumerate}
  \item Related works.
    \only<article>{%
      This section should give an overview of the existing research literature 
      related to \cref{mainRQ}.
      That is, you should find all papers that answers some aspect of it.
      You should note what question each paper answers and summarize how they 
      do that.
      (This focuses on \cref{LOevaluate}.)%
    }%
    \only<presentation>{%
      \begin{itemize}
        \item Find all papers answering 
          some aspect.
        \item What (sub)question does each paper answer and how?
      \end{itemize}
    }
  \item Missing aspects.
    \only<article>{%
      In this section, you pose questions that are also related to 
      \cref{mainRQ}, but has not been covered by the literature.
      For each such question, you must propose a method that correctly provides 
      an answer\footnote{%
        But remember, you don't have to actually perform any of these 
        methods.
      }.
      You must also discuss why this method is suitable to answer the question 
      and any limitations.
      (This focuses on \cref{LOapply,LOevaluate}.)%
    }%
    \only<presentation>{%
      \begin{itemize}
        \item State relevant (sub)questions not asked by the literature.
          Propose a method to answer each.
        \item \alert<+>{Discuss why the method is suitable and any 
          limitations.}
      \end{itemize}
    }
  \item Conclusion.
    \only<article>{%
      This section ties the sack.
      Here you connect the questions (yours and from related works) and the 
      types of answers gained (through the methods) and piece them back into 
      \cref{mainRQ}.
      You also summarize how well you find \cref{mainRQ} to be answered, if 
      there are any \enquote{holes that need filling}.
      (This focuses on \cref{LOrelate}.)%
    }%
    \only<presentation>{%
      \begin{itemize}
        \item Connect the questions and answers back to \cref{mainRQ}.
        \item \alert<+>{Discuss how well the main RQ is answered.}
      \end{itemize}
    }
%  \item your original plan for the course work, adaptations made and what you 
%    learned.
\end{enumerate}
\end{frame}
This assesses \cref{LOrelate,LOevaluate,LOapply,LOcomm}.

%\paragraph{Feedback you should provide}
%
%You should also
%\begin{itemize}
%  \item review someone else's report.
%\end{itemize}
%This assesses \cref{LOrelate,LOevaluate} and provides a learning opportunity 
%for \cref{LOcomm}.
%This also contributed to \cref{LOplan} as one would learn from others' 
%mistakes.
%
%The presentation should also include
%\begin{itemize}
%  \item what you learned from the feedback of the reviewer, and
%  \item what you learned from the work you reviewed.
%\end{itemize}
%This also assesses \cref{LOcomm}.

\subsection{Assessment criteria}

To assess the learning objectives
(\cref{LOrelate,LOevaluate,LOapply,LOcomm,LOplan})
we use the following criteria.
These criteria are also included as a rubric in the assignment where you hand 
in your report.

We note, however, that there is no criteria for \cref{LOplan}.
This is because we do not assess the plan itself, but rather the fact that you 
finished on time.

You'll need a pass on all criteria to pass the assignment and the course.

\begin{frame}<presentation>
  \begin{block}{Assessment criteria}
    \begin{itemize}
      \item \alert<+>\LOrelate
      \item \LOapply
      \item \alert<+>\LOevaluate
      \item \LOcomm
    \end{itemize}
  \end{block}
\end{frame}
{\RaggedRight
\begin{fullwidth}
\begin{longtable}
{p{0.20\columnwidth}p{0.20\columnwidth}p{0.20\columnwidth}p{0.20\columnwidth}}
\toprule
\textbf{Learning objective}
  & \textbf{Criteria}
  & \textbf{Pass}
  & \textbf{Fail}
  \\*
\midrule
\endhead
\only<article>{\cref{LOrelate}:}
The student is able to \LOrelate
  & The main research question is explored from relevant aspects?
  & There might be more aspects to explore, but the most important ones are 
  covered. Motivate why no more aspect need to be explored.
  & There is at least one aspect missing that can be motivated to be 
  important. Motivate which one.
  For instance, do we need to ask another more detailed (research) question 
  to be able to answer the main research question in a meaningful way? Do 
  they address the question from just a single perspective?
  \\*
\only<article>{\cref{LOapply}:}
The student is able to \LOapply
  & The methods are suitable to answer the questions?
  & All questions have suggested methods that can actually answer the 
  question correctly.
  Motivate why this is the case.
  & There is at least one question that will not be answered correctly with 
  the suggested method.
  State which one and why.
  For instance, the method might only answer part of the question.
  %Or not at all.
  \\*
\only<article>{\cref{LOevaluate}:}
The student is able to \LOevaluate
  & Are all quality aspects considered in the discussion?
  & The most important quality aspects are considered and discussed.
  & At least one important quality aspect is missing.
  State which one and motivate why it's important enough that it must be 
  treated.
  \\*
  & Are all ethical aspects considered?
  & The most important ethical aspects are considered and discussed.
  & At least one important ethical aspect is missing.
  State which one and motivate why it's important enough that it must be 
  treated.
  \\*
\only<article>{\cref{LOcomm}:}
The student is able to \LOcomm
  & Is the report written as short as possible?
  & The report can probably be slightly shortened, but not by much.
  & The report can be shortened considerably.
  Give at least one example of where and how.
  \\*
  & Is the report clear and easy to understand?
  & The report is easy to understand.
  & Some parts of the report must be read more than once to understand.
  (Or worse.)
  Give at least one example.
  \\*
  & Are the arguments clearly stated and well motivated?
  & All arguments are clearly stated and well motivated.
  & At least one argument is not clearly stated or not well motivated.
  State which one and motivate why it's not clear or well motivated.
\end{longtable}
\end{fullwidth}
}

\subsection<article>{Plagiarism}

You work in the groups that you've signed up for.
You may discuss with others, search the literature and use tools such as 
ChatGPT as long as you are transparent with its use:
\ie you cite sources and state what you've used ChatGPT for\footnote{%
  For instance, if you've asked it for suggestions on how to shorten a piece of 
  text, you state that you did and how you used its suggestion.
}.

The governing rules that apply are from Chapter 10 of the Higher Education 
Ordinance, from its \S~1:
\begin{quote}
  Disciplinary measures may be taken against students who
  \begin{enumerate}
    \item use prohibited aids or other methods to \emph{attempt to deceive} 
      \textins{my emphasis} during examinations or other forms of assessment of 
      study performance,
  \end{enumerate}
\end{quote}

Not mentioning that you've used ChatGPT or discussed the topic with people 
outside the group is considered an attempt to deceive.

\section{The final seminar}

During the final seminar we will discuss the questions and methods that you've 
covered in the report and how well \cref{mainRQ} was covered.

For the seminar you should prepare slides.
Each slide should focus on a research question.
If the research question was covered by the literature, you should have a 
reference to the paper on the slide.

During the seminar, with the slide you should be able to explain the question 
and your evaluation of how well the method answers the question.
(You can use more than one slide if it improves the presentation, but it 
shouldn't be necessary.)

The participants will take turn in presenting the question, method and 
evaluation.
The we discuss jointly.
We will want to discuss closely related questions, try to group them by that in 
the slide deck.

\begin{frame}<presentation>
  \begin{activity}[Order]
    \begin{itemize}
      \item What order to present?
      \item You'll have one minute to write \enquote{me!}.
      \item The last one to write \enquote{me!} goes first.
    \end{itemize}
  \end{activity}
\end{frame}

\begin{frame}<presentation>
  \mainRQ*
  \begin{activity}
    \begin{itemize}
      \item The presenter's favourite aspect/subquestion first.
      \item Summarize the question and the method\footnote{%
          Cite if from a paper.
        }.
      \item Discuss why you think this is good\footnote{%
          Focuses on the weak points pointed out above.
        }:
        \begin{itemize}
          \item Why does this method answer the question?
          \item Why does the answer to the question help answering the main RQ.
          \item What ethical aspects are there to consider?
        \end{itemize}
      \item What do the others think?
    \end{itemize}
  \end{activity}
\end{frame}
